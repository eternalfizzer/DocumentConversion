%% master tex file for generating Information Security Policy
% Renee Boileau for Bryan Thornton
% last updated 2017-05-04

% *** Edit inputs in header.tex only ***

\documentclass[sec]{policy}

% sample Policy header file

\orgname{Some Good Organization} % 
\reldate{May 5, 2017} % NEW

% Overview
\chartfile{./graphics/chart.png} % filename with path
\totRiskLevel{Medium} % Low, Medium, High
\totRiskScore{61} % []0-100%]

% Information Security Fundamentals table
\fOne{Deficient} % Deficient, Passing
\fTwo{Deficient}
\fThree{Passing}
\fFour{Passing}
\fFive{Deficient}
\fSix{Deficient}
\fSeven{Passing}
\fEight{Deficient}
\fNine{Passing}
\fTen{Passing}
\fEleven{Passing}
\fTwelve{Passing}
\fThirteen{Passing}
\fFourteen{Deficient}
\fFifteen{Passing}

% Focus Area 01 table
\areaOneRiskLevel{Medium} % Low, Medium, High
\areaOneScore{65} % [%] 0-100
\areaOneRec{Consider formalizing your risk assessment framework.} % pipe-separated list

% Focus Area 02 table
\areaTwoRiskLevel{High}
\areaTwoScore{33}
\areaTwoRec{Adopt a formal Information Security Policy |
	Adopt a formal Acceptable Use Policy
	}%

% Focus Area 03 table
\areaThreeRiskLevel{High}
\areaThreeScore{47}
\areaThreeRec{Formalize  your individual pieces of a program into a formal Information Security Program |
	Consider implementing a formal update schedule for your organizational policies |
	Consider having an outside third party review your organization's Information Security posture on a periodic basis |
	The methods that vendors and third parties use to protect your data are as important as the method and controls you use.  Assess your vendors' Information Security programs before conducting business |
	Consider making the protection of your organization's information a part of your vendor contracts
	}%

% Focus Area 04 table
\areaFourRiskLevel{High}
\areaFourScore{5}
\areaFourRec{Consider formalizing your Data Classification Guidelines |
	Organizational assets and information should be clearly labeled and identified according to your Data Classification Guidelines
	} %

% Focus Area 05 table
\areaFiveRiskLevel{tbd}
\areaFiveScore{0}
\areaFiveRec{tbd}

% Focus Area 06 table
\areaSixRiskLevel{tbd}
\areaSixScore{0}
\areaSixRec{tbd}

% Focus Area 07 table
\areaSevenRiskLevel{Medium}
\areaSevenScore{74}
\areaSevenRec{Careless disposal of information is a leading cause of data breaches.  Make sure that all media and information is securely destroyed before disposal or reuse |
	Consider setting your antivirus solution to update and run at least once per week on your workstations. |
	Consider implementing formal change control procedures to evaluate and if necessary roll back changes made to the organization's systems. |
	Consider archiving audit logs for all user activity as appropriate.  Audit logs make both troubleshooting and investigations easier and less time consuming. |
	Assign someone to review your audit logs on a periodic basis.  Many instances of fraud have been uncovered by a casual glance at log files. |
	Test your backup media from time to time, especially if you are using cassette media.  Backup media has a finite lifespan and must be replaced periodically. |
	Consider whether you really need sensitive personal information to be stored on laptop computers.  In cases where the answer is yes, continue with whole disk encryption. |
	Consider whether or not business applications really require sensitive personal information to be collected via the web.  In cases where the answer is yes, ensure that you continue the use of appropriate encryption. |
	Consider formally testing your websites for common vulnerabilities on a periodic basis. |
	Consider whether or not your business really needs to store credit card information.  In cases where the answer is yes, continue the use of appropriate encryption.
	} %

% Focus Area 08 table
\areaEightRiskLevel{tbd}
\areaEightScore{0}
\areaEightRec{tbd}

% Focus Area 09 table
\areaNineRiskLevel{tbd}
\areaNineScore{0}
\areaNineRec{tbd}

% Focus Area 10 table
\areaTenRiskLevel{tbd}
\areaTenScore{0}
\areaTenRec{tbd}

% Focus Area 11 table
\areaElevenRiskLevel{tbd}
\areaElevenScore{0}
\areaElevenRec{tbd}

% Focus Area 12 table
\areaTwelveRiskLevel{tbd}
\areaTwelveScore{0}
\areaTwelveRec{tbd}

% Next Steps list
\nextSteps{Stop to smell the roses. |
	Call your mother. |
	Do the next thing on your bucket list.
	}
 % custom inputs file

\begin{document}

	\maketitle

	% page 1
	\thispagestyle{plain}
	\titleline
	
	\doctable
	\tableofcontents
	
%	\vfill
	\section{Executive Summary}
	
		The \thetitle\ exists in order to provide the organization’s staff with a current set of clear and concise information security policies.  These policies provide direction for the appropriate protection of the organization’s information and assets.
	
		The \thetitle\ has been created as a component of an overall Information Security Program (ISP) for the organization.  The ISP outlines the organization’s mission and objectives as they relate to information security, outlines details that are responsible for information security, documents policies relating to information security, indicates how the program is to be communicated and how people in the organization must be trained on their responsibilities, and includes a roadmap of how the program is to be carried out.  In addition, the program includes strategies for its ongoing evaluation and adjustment, addressing of compliance issues, and management reporting.
		
	\section{Purpose and Guiding Principles}
		
		The purpose of this policy is to provide general guidance and specific recommendations for the protection of \theOrganization\ information technology resources and the information stored on those resources.  Additionally information and assets that exist in hard form are also protected by this policy.  These information security measures are intended to protect the organization’s information and assets and to preserve the privacy of \theOrganization\ ’s employees, contractors, vendors, and other related third parties.
		
		The broad goal of information security at \theOrganization\ is to maintain Confidentiality, Integrity, and Availability of data.  To achieve this goal, \theOrganization\ has identified a set of core security principles to guide the creation of a policy based on the ISO 17799 framework.  The policy will, in turn, be supported by detailed operational procedures.  These simple principles make up the foundations of a strong security posture.
		
		\begin{description}
			\item[Universal Participation –] Every component of an organization could be a potential avenue of entry for unauthorized intruders.  Thus a strong security infrastructure requires the cooperation of all parties in the organization.  Everyone is responsible for security.
			\item [Risk-Based security –] An organization’s security is defined by the unique risks it faces.  These risks should be identified regularly and should remain the primary focus of any security policy or program.
			\item [Deny All That is Not Explicitly Permitted –] Anything not explicitly allowed is denied.
			\item [Least-Privilege –] Users and systems should only have minimum level of access necessary to perform their defined function.  All unnecessary levels of access should be restricted unless explicitly needed.
			\item [Defense-in-Depth –] Overall security should not be reliant upon a single defense mechanism.  If an outer security perimeter is penetrated, underlying layers should be available to resist the attack.
			\item [Compartmentalization –] If one compartment is compromised, it should be equally difficult for an intruder to obtain access to each subsequent compartment.
			\item [Secure Failure –] When a system’s confidentiality, integrity, or availability is compromised, the system should fail to a secure state.
			\item [Defense through Simplicity –] A simple system is more easily secured than a complex system, as there is a reduced chance for error.
			\item [Dedicated Function –] Systems should be single-purposed to avoid potential conflicts or redundancies that could result in security exposures.
			\item [Need-to-Know –] Information will only be circulated to those parties that require it in order to perform their defined business function.
			\item [Effective Authentication and Authorization –] Firmly established identity and role-based authorization are essential to making informed access control decisions.
			\item [Audit Integrity –] Audit log events that are generated may not be altered by the entity that generated the event.
		\end{description}
		
	\section{Scope}
	
		This policy applies to all divisions of \theOrganization\.  It covers all \theOrganization\ information technology resources, information that is or may be stored in digital form, as well as information and assets that may exist in physical form.  All creation, processing, communication, storage, distribution and disposal of \theOrganization\ information and assets are covered by this policy.  Each employee of \theOrganization\, contractor and other related third parties are bound by the guiding principles, statement of policy and related procedures outlined in this policy. 
		
	\section{Statement of Policy}
		
		The Information Security Policy exists in order to provide the organizations staff with a current set of clear and concise information security policies.  These policies provide direction for the appropriate protection of the organization’s information and assets.
		
		The Information Security Policy has been created as a component of an overall Information Security Program (“ISP”) for the organization.  The ISP outlines the organization’s mission and objectives as they relate to information security, outlines details that are responsible for information security, documents policies relating to information security, indicates how the program is to be communicated and how people in the organization must be trained on their responsibilities, and includes a roadmap of how the program is to be carried out.  In addition, the program includes strategies for its ongoing evaluation and adjustment, addressing of compliance issues, and management reporting.
		
		The Information Security Policy has been reviewed, approved, and is endorsed by \theOrganization\ management.
		
		The Information Security Policy applies to all \theOrganization\ employees, contractors, and any third-party providers that support any of the \theOrganization\'s services.
		
		The Information Security Policy document contains rules and requirements that must be met in the delivery and operation of the \theOrganization\'s services.  More detailed standards and specific procedures must be developed as adjuncts to this Information Security Policy to provide implementation level details for carrying out specific operational tasks.  The procedures must be the instrument by which these \theOrganization\ Security Policies are converted into action.
		
		The Information Security Policy must be located in a central repository that is accessible to all \theOrganization\ employees and related third parties.
		
		The Information Security Policy must be distributed to all new and existing \theOrganization\ employees for review.  All \theOrganization\ employees, contractors and third party providers are required to sign an agreement representing the fact that they have reviewed, and agree to adhere to, all policies within the Information Security Policy document.
		
		Exceptions to the Information Security Policy must be authorized by both \theOrganization\ management and the affected asset owner.  Please refer to «InformationSecurityResponsibleParty» for exception process details.
		
		\section{Procedures}
		
			Within this Section, the phrases “must” and “recommended” have specific meanings where highlighted in boldface. If a department correctly adheres to the guidelines given as “must”, then it can be considered as meeting the requirements for this policy. If they also adhere to the guidelines given as “recommended”, then they can be considered to be meeting the minimum requirements to be in accordance with generally accepted information security practices.
			
			\begin{enumerate}[label = \Alph*.]
				\item Departments must limit access to information and assets to individuals that have been explicitly authorized by management.
				\item Departments must protect all information resources (e.g. computers, communications, software, data) from theft, tampering, misuse, malicious software (e.g. viruses, worms, Trojan Horses), destruction, and loss.
				\item Managers must ensure that individuals who use \theOrganization\ information resources or assets have read both the \theOrganization\ Information Security Policy as well as the \theOrganization\ Acceptable Use Policy. Employees are bound by these policies as conditions of their employment.  It is recommended that these policies be reviewed with individuals on at least an annual basis.
				\item The IT manager must ensure individual and organizational accountability for the use and protection of information systems through the use of unique identifiers and authentication methods (e.g. user ID/password, digital certificates, or biometrics). 
				\item Departments and managers must prohibit the sharing or unauthorized disclosure of passwords or other confidential access information (e.g. access phone numbers, account names, codes, network information) except as explicitly authorized.
				\item Each department must provide the IT manager prompt notification of changes in employee status (e.g. transfers, terminations, retirement) for all employees and other users of IT resources and systems.
				\item Each department must control access to information based on the confidentiality of the data and the requirements established in the Data Access Policy. (NOTE: this requires that all significant collections of information have a defined data steward who is responsible for overseeing the appropriate use of the data collection).
				\item The IT manager must ensure that appropriate controls exist to ensure only authorized use of software and hardware that is designed for bypassing or breaking through information security measures and procedures. Ensure appropriate audit procedures are established to monitor for inappropriate use of software and hardware that is capable of bypassing or breaking through information security measures and procedures.
				\item As necessary the IT manager must be capable of producing, reviewing, and retaining audit trails of security related information for all systems that process or store \theOrganization\ information.
				\item It is recommended that each department or group regularly perform self-assessments and/or audits to detect security vulnerabilities and non-compliance to \theOrganization\' security policies and procedures. Upon discovery, each department must initiate corrective actions to ensure that compliance with these policies and procedures is restored. 
				\item Apply cryptographic controls as appropriate to protect critical information that is transmitted or stored on \theOrganization\'s network, workstations, laptops or servers.  Passwords, keys, certificates or other means to decrypt data must be provided to management for escrow to facilitate the recovery of data.  All laptop computers must use encryption to protect stored data in case of loss.
				\item Each department as well as the IT manager must ensure that and procured IT resources are have sufficient capabilities to allow compliance with this policy, such as User ID/passwords, system integrity protection, and audit logging.
				\item Each department must make sufficient plans to maintain continued availability of business-critical resources and information through appropriate business continuity and disaster recovery planning.  It is recommended that departments review and test these plans on at least an annual basis.
				\item Each department must apply information retention procedures that satisfy external and internal requirements, while ensuring the application of procedures to dispose of information that is no longer needed.
			\end{enumerate}
			
		\section{Roles and Responsibilities}
		
			\subsection{Senior Management}
			
				\theOrganization\ Senior management is responsible for:
				
				\begin{enumerate}[label = \Alph*.]
					\item Promulgating and enforcing the policies, standards, procedures, and guidelines for the protection of IT resources and information.
					\item Furnishing necessary funding and other resources or limiting and eliminating services to ensure continued compliance with this policy. 
					\item Appointing an Information Security Coordinator and/or establishing departmental computer support and system administrators. Providing appropriate training and resources to the person(s) responsible for information security-related tasks.
					\item Specifying and applying sanctions consistent with Human Resources policies to individuals and divisions that break provisions of this policy, either willfully, accidentally, or through ignorance.
					\item Designating Data Stewards for each significant collection of business information, who in turn are responsible for determining the value of their information and implementing appropriate security measures as specified in the Data Access Policy.
					\item Sponsoring internal awareness and training programs to familiarize employees, contractors and third-party providers with the security policy, procedures and recommended practices.
					\item Defining guidelines and intervals for the review and update of this policy and to reassess existing risks and to identify potential new risks to \theOrganization\ assets and information.
				\end{enumerate}
				
			\subsection{Employees}
				
				Each \theOrganization\ employee is responsible for understanding and complying with the policies and procedures relating to information technology security and for fully cooperating with the information security staff at all levels to protect \theOrganization\'s information and assets.
				
				Each employee must become familiar with \theOrganization\'s Computer and Network Usage Policy.
				
				\theOrganization\ computer and communications systems must be used for business purposes only. Incidental personal use is permissible if the use (a) does not consume more that a trivial amount of resources that could otherwise be used for business purposes, (b) does not interfere with worker productivity, and (c) does not preempt any business activity. Examples of permissible incidental use include – the occasional use of electronic mail (email) or web access for other than official purposes.
				
				Using \theOrganization\ systems to download, use, or re-distribute unlicensed or inappropriate software, copyrighted movies, copyrighted music, or pornographic materials, place the Institute at risk and will not be tolerated. Conduct in violation of this policy may result in sanctions as provided in the Computer and Network Usage Policy. Report all actual or suspected instances of security or policy violations in accordance with the Incident Reporting section of this policy.
				
		\section{Compliance}
				
			Any person who uses \theOrganization\'s information or assets consents to all provisions of this policy and agrees to comply with all of its terms and conditions, as well as with relevant state and federal laws and regulations.  Users have a responsibility to use these resources in an effective, ethical and lawful manner.  Any violation of this policy may result in disciplinary or administrative sanctions including loss of privileges, monitoring of use and up to and including termination depending on the severity and intent of offense.  Additionally, non-compliance with this policy resulting in loss or disclosure of data may result in personal civil and/or criminal liability.
				
		\nobreaksection{Policy Modifications}
				
			This policy may be changed by \theOrganization\ Senior Management at any time, but typically will be modified in response to newly identified threats or risks.  Changes to this policy will be communicated and distributed to all affected parties.  Most major changes to the policy will be made during official policy review sessions on an annual basis, but if required a policy review session may be convened on a special basis.
				
		\nobreaksection{Communication}
				
			Upon approval, this policy is to be distributed to all \theOrganization\ employees, contractors, vendors and related third parties.  Upon subsequent revisions, updates or amendments to this policy affected individuals will be notified of the change along with an office or individual to whom they can direct additional questions.
				
		\section{Acknowledgement of Understanding}
		
			I have been provided with a copy of the \theOrganization\ Information Security Policy and have read and reviewed the policy.  I have been provided the opportunity to ask questions about the policy and am aware of resources to which I may look for more information.  I understand that it is my responsibility to comply with the provisions of the policy and that failure to do so is subject to disciplinary action up to and including immediate termination.  My signature below indicated that I have been provided with and have read the \theOrganization\ Information Security Policy and furthermore indicates my understanding and willingness to comply with the provisions of the policy.
		
			\signblock
			
\end{document}